\subsection{\it General features}
Soker \& Dgani (1997) conduct a theoretical study of the processes involved when
the ISM magnetic field is important in the interaction.
In the case where the ISM is fully ionized, we define four characterizing
velocities of the interaction process:
the adiabatic sound speed $v_s=(\gamma kT/\mu m_H)$ and the Alfven
velocity $v_A = B_0/(4 \pi \rho_0)^{1/2}$ of the ISM, the expansion
velocity of the nebula $v_e$, and the relative velocity of the PN
central star and the ISM $v_\ast$.
The interesting cases, with the magnetic field lines being at a large angle
to the relative velocity direction, are: \\

\begin{enumerate}
\item $v_\ast \gg v_A \sim v_s \sim v_e$, and a rapid cooling behind the
shock wave.
Both the thermal and magnetic pressure increase substantially behind
a strong shock. If radiative cooling is rapid, however, the magnetic pressure
will eventually substantially exceed the thermal pressure, leading to
several strong MHD instabilities around the nebula, and probably to
magnetic field reconnection behind the nebula.

\item $v_\ast \gg v_A \sim v_s \sim v_e$, and negligible cooling behind the
shock. The thermal pressure, which grows more than the magnetic pressure in a
strong shock, will dominate behind the shock.
 Magnetic field reconnection is not likely to occur behind the nebula.
This domain characterizes the interaction of the solar wind with
the atmospheres of Venus and Mars (e.g., Phillips \& McComas 1991).
\end{enumerate}